%%
%% This is file `sample-sigconf-authordraft.tex',
%% generated with the docstrip utility.
%%
%% The original source files were:
%%
%% samples.dtx  (with options: `all,proceedings,bibtex,authordraft')
%% 
%% IMPORTANT NOTICE:
%% 
%% For the copyright see the source file.
%% 
%% Any modified versions of this file must be renamed
%% with new filenames distinct from sample-sigconf-authordraft.tex.
%% 
%% For distribution of the original source see the terms
%% for copying and modification in the file samples.dtx.
%% 
%% This generated file may be distributed as long as the
%% original source files, as listed above, are part of the
%% same distribution. (The sources need not necessarily be
%% in the same archive or directory.)
%%
%%
%% Commands for TeXCount
%TC:macro \cite [option:text,text]
%TC:macro \citep [option:text,text]
%TC:macro \citet [option:text,text]
%TC:envir table 0 1
%TC:envir table* 0 1
%TC:envir tabular [ignore] word
%TC:envir displaymath 0 word
%TC:envir math 0 word
%TC:envir comment 0 0
%%
%%
%% The first command in your LaTeX source must be the \documentclass
%% command.
%%
%% For submission and review of your manuscript please change the
%% command to \documentclass[manuscript, screen, review]{acmart}.
%%
%% When submitting camera ready or to TAPS, please change the command
%% to \documentclass[sigconf]{acmart} or whichever template is required
%% for your publication.
%%
%%
\documentclass[sigconf,authordraft]{acmart}

%%
%% \BibTeX command to typeset BibTeX logo in the docs
\AtBeginDocument{%
  \providecommand\BibTeX{{%
    Bib\TeX}}}

%% Rights management information.  This information is sent to you
%% when you complete the rights form.  These commands have SAMPLE
%% values in them; it is your responsibility as an author to replace
%% the commands and values with those provided to you when you
%% complete the rights form.
\setcopyright{acmlicensed}
\copyrightyear{2018}
\acmYear{2018}
\acmDOI{XXXXXXX.XXXXXXX}

%% These commands are for a PROCEEDINGS abstract or paper.
\acmConference[Conference acronym 'XX]{Make sure to enter the correct
  conference title from your rights confirmation emai}{June 03--05,
  2018}{Woodstock, NY}
%%
%%  Uncomment \acmBooktitle if the title of the proceedings is different
%%  from ``Proceedings of ...''!
%%
%%\acmBooktitle{Woodstock '18: ACM Symposium on Neural Gaze Detection,
%%  June 03--05, 2018, Woodstock, NY}
\acmISBN{978-1-4503-XXXX-X/18/06}


%%
%% Submission ID.
%% Use this when submitting an article to a sponsored event. You'll
%% receive a unique submission ID from the organizers
%% of the event, and this ID should be used as the parameter to this command.
%%\acmSubmissionID{123-A56-BU3}

%%
%% For managing citations, it is recommended to use bibliography
%% files in BibTeX format.
%%
%% You can then either use BibTeX with the ACM-Reference-Format style,
%% or BibLaTeX with the acmnumeric or acmauthoryear sytles, that include
%% support for advanced citation of software artefact from the
%% biblatex-software package, also separately available on CTAN.
%%
%% Look at the sample-*-biblatex.tex files for templates showcasing
%% the biblatex styles.
%%

%%
%% The majority of ACM publications use numbered citations and
%% references.  The command \citestyle{authoryear} switches to the
%% "author year" style.
%%
%% If you are preparing content for an event
%% sponsored by ACM SIGGRAPH, you must use the "author year" style of
%% citations and references.
%% Uncommenting
%% the next command will enable that style.
%%\citestyle{acmauthoryear}


%%
%% end of the preamble, start of the body of the document source.
\begin{document}

%%
%% The "title" command has an optional parameter,
%% allowing the author to define a "short title" to be used in page headers.
\title{Using LLM for code generation and repair in functional programming assessments: challenges and potential
}

%%
%% The "author" command and its associated commands are used to define
%% the authors and their affiliations.
%% Of note is the shared affiliation of the first two authors, and the
%% "authornote" and "authornotemark" commands
%% used to denote shared contribution to the research.
% \author{Ben Trovato}
% \authornote{Both authors contributed equally to this research.}
% \email{trovato@corporation.com}
% \orcid{1234-5678-9012}
% \author{G.K.M. Tobin}
% \authornotemark[1]
% \email{webmaster@marysville-ohio.com}
% \affiliation{%
%   \institution{Institute for Clarity in Documentation}
%   \city{Dublin}
%   \state{Ohio}
%   \country{USA}
% }


%%
%% By default, the full list of authors will be used in the page
%% headers. Often, this list is too long, and will overlap
%% other information printed in the page headers. This command allows
%% the author to define a more concise list
%% of authors' names for this purpose.
\renewcommand{\shortauthors}{Trovato et al.}

%%
%% The abstract is a short summary of the work to be presented in the
%% article.
\begin{abstract}

% While there already exists a line of work evaluating the performance of LLMs such as ChatGPT in solving independent programming tasks, few of them investigate how the model could behave in solving programming assignments/homework that is designed to fit students' learning trajectories. 

The recent introduction of ChatGPT has drawn significant attention
from both industry and academia due to its impressive capabilities in
solving a diverse range of tasks, including language translation, text
summarization, and computer programming.
% Unlike previous large language models, ChatGPT effectively bridges
% the gap between human and AI performance in multiple key domains. 
Its capability for writing, modifying, and even
correcting code together with its ease of use and access is already 
dramatically impacting computer science education. 
%To comprehensively evaluate such impact, it is essential to assess
%its ability to solve programming tasks in the context of computer
%science education. 
This paper aims to explore how well ChatGPT can perform in an
introductory-level functional language programming course.  Our comprehensive evaluation provides valuable
insights into ChatGPT's impact from both student and instructor
perspectives. Additionally, we identify several potential benefits
that ChatGPT can offer to both groups. Overall, we believe that this
study significantly clarifies and advances our understanding of
ChatGPT's capabilities and potential impact on computer science education.
\end{abstract}

%%
%% The code below is generated by the tool at http://dl.acm.org/ccs.cfm.
%% Please copy and paste the code instead of the example below.
%%
% \begin{CCSXML}
% <ccs2012>
%  <concept>
%   <concept_id>00000000.0000000.0000000</concept_id>
%   <concept_desc>Do Not Use This Code, Generate the Correct Terms for Your Paper</concept_desc>
%   <concept_significance>500</concept_significance>
%  </concept>
%  <concept>
%   <concept_id>00000000.00000000.00000000</concept_id>
%   <concept_desc>Do Not Use This Code, Generate the Correct Terms for Your Paper</concept_desc>
%   <concept_significance>300</concept_significance>
%  </concept>
%  <concept>
%   <concept_id>00000000.00000000.00000000</concept_id>
%   <concept_desc>Do Not Use This Code, Generate the Correct Terms for Your Paper</concept_desc>
%   <concept_significance>100</concept_significance>
%  </concept>
%  <concept>
%   <concept_id>00000000.00000000.00000000</concept_id>
%   <concept_desc>Do Not Use This Code, Generate the Correct Terms for Your Paper</concept_desc>
%   <concept_significance>100</concept_significance>
%  </concept>
% </ccs2012>
% \end{CCSXML}

\ccsdesc[500]{Do Not Use This Code~Generate the Correct Terms for Your Paper}
\ccsdesc[300]{Do Not Use This Code~Generate the Correct Terms for Your Paper}
\ccsdesc{Do Not Use This Code~Generate the Correct Terms for Your Paper}
\ccsdesc[100]{Do Not Use This Code~Generate the Correct Terms for Your Paper}

%%
%% Keywords. The author(s) should pick words that accurately describe
%% the work being presented. Separate the keywords with commas.
% \keywords{Do, Not, Us, This, Code, Put, the, Correct, Terms, for,
%   Your, Paper}
%% A "teaser" image appears between the author and affiliation
%% information and the body of the document, and typically spans the
%% page.
% \begin{teaserfigure}
%   \includegraphics[width=\textwidth]{sampleteaser}
%   \caption{Seattle Mariners at Spring Training, 2010.}
%   \Description{Enjoying the baseball game from the third-base
%   seats. Ichiro Suzuki preparing to bat.}
%   \label{fig:teaser}
% \end{teaserfigure}

\received{20 February 2007}
\received[revised]{12 March 2009}
\received[accepted]{5 June 2009}

%%
%% This command processes the author and affiliation and title
%% information and builds the first part of the formatted document.
\maketitle

\section{Introduction}

Large Language Models (LLMs) have made significant strides in various domains, including natural language processing, machine translation, and text summarization. Among these applications, the use of LLMs for program synthesis and generation has been particularly noteworthy. Projects such as OpenAI's Codex and DeepMind's AlphaCode have demonstrated the impressive ability of LLMs to generate code, showcasing their potential in automating programming tasks. Recent studies have extended this application to programming education. A experiment by \cite{6:3:380} reveals that learners with access to Codex during the training phase performed slightly better on the evaluation.
The work by \cite{1555162} used the latest versions of ChatGPT, GitHub Copilot, and Amazon CodeWhisperer to generate Python code using the benchmark HumanEval Dataset, and they achieved 65.2\%, 46.3\%, and 31.1\% of the time, respectively. A study by \cite{2004:ITE:1009386.1010128} investigates how a pedagogically-designed LLM-based chatbot supports students’ debugging efforts in an introductory programming course. Their data revealed that students appreciated the content and experiential knowledge provided by the chatbot, but did not view it as a primary source for learning debugging strategies. Additionally, the research by \cite{6:1:1} used LLM not only for generating code, but also for fixing buggy codes. They found that LLMs tend to incorrectly use existing types or hallucinate entirely new types.



However, the related research concerns particularly commonly used programming languages like Java and Python, which leads to a notable gap in research regarding the application of LLMs to statically typed functional programming languages, such as OCaml, which are often used in advanced programming courses. This study aims to address this gap by evaluating and comparing LLM performance in the context of OCaml programming education. Specifically, we investigate whether LLMs can solve OCaml programming assignments, and effectively repair faulty OCaml code. The significance of this research lies in its potential to provide insights into LLM capabilities in typed functional programming, allow comparison of different LLMs' performance in this paradigm, inform educators on effective LLM use in OCaml coursework, and guide the design of programming exercises that promote learning while maintaining academic integrity.



Evaluating LLMs poses several challenges. We often lack knowledge about their training data, raising questions about the reproducibility of results. Additionally, the variability in LLM outputs can lead to inconsistent results, complicating the evaluation process. Moreover, equity and inclusion concerns arise, as the cost of evaluating LLMs may be prohibitive for some reviewers or instructors. Despite these challenges, our research aims to shed light on the practical and conceptual aspects of using LLMs for programming education, particularly in statically typed functional programming languages. By addressing these questions, we hope to contribute to the growing body of knowledge on LLMs and their potential to transform programming education.

\section{Related work}

2.1 GPT-4
- Overview of GPT-4's capabilities
- Studies on GPT-4's performance in code generation
- Specific applications in programming education
2.2 GitHub Copilot
- Introduction to Copilot and its underlying technology
- Evaluation of Copilot's effectiveness in code completion and generation
- Impact on developer productivity and learning
2.3 Code Llama
- Overview of Code Llama and its specialized training
- Comparative studies with other code-focused LLMs
- Unique features relevant to programming tasks
2.4 comparison of the thress

\section{Method}


% eval method
% To evaluate how effectively LLMs can fix errors, we conducted experiments using various LLMs in different scenarios: zero-shot and one-shot. In the zero-shot scenario, we provided the LLM with a faulty program and asked it to generate a corrected version without any prior examples. In the one-shot scenario, we first provided the LLM with a faulty program and its corrected version. Then, we presented another similar faulty program and asked the LLM to generate the corrected version.

% We identified similar programs by selecting faulty programs from a student's timeline, considering submissions either earlier or later in the course. This approach helped us evaluate the LLMs' ability to generalize error correction across different contexts.

For syntax errors, we randomly sampled 500 programs from student submissions, equally distributed among ten homework assignments (HW1-10) from Fall 2022. 

% We used two modes for this evaluation: Mode A, where the prompt explicitly mentioned that the program contained a syntax error, and Mode B, where the prompt did not specify the type of error. 

We compared the results from Fall 2022 to those from Winter 2023 to validate the LLMs' performance.

We used the same sampling method for type errors and also looked at "hard stuck states" to test the LLMs on tougher errors. Logical errors were tested using the same sampling approach to see how well LLMs handle complex errors in student code.

We also evaluated the LLMs' capability to write complete programs by providing them with 10 different programming assignments and assessing their performance. We measured the pass rate as the percentage of assignments where the LLM generated a correct solution that passed all tests on the first try. Additionally, we evaluated the success rate over multiple attempts (k = 1, 5, 10), recording the number of full grades achieved across these attempts.

We used two different modes for this evaluation: no feedback, where the LLM generated solutions without any intermediate feedback, and compiler feedback, where the LLM received one round of feedback from a compiler, allowing it to refine its solution before the final evaluation. For each program generated by the LLM, we recorded the highest score achieved over the three attempts. This comprehensive evaluation helped us understand the LLMs' potential in generating correct and high-quality code.

By conducting these evaluations, we aimed to gain insights into the strengths and limitations of different LLMs in both error correction and program generation tasks, particularly in the context of teaching statically typed functional programming languages like OCaml.
\section{Evaluation}

\section{Threats To Validity}
In our evaluation, we assume that the LLMs are already fully trained and that no additional re-training or learning occurs during the evaluation process. This assumption is crucial because it impacts the validity and reproducibility of our results. If any form of additional learning were to take place during the evaluation, it could skew the outcomes and lead to an overestimation of the LLMs' capabilities.

Furthermore, the variability in LLM outputs poses a challenge to the consistency of our findings. Since LLMs can produce different results given the same input on different occasions, this variability must be accounted for in our analysis. We mitigate this by averaging results over multiple runs, but inherent unpredictability remains a concern.

Another threat to validation arises from the unknown nature of the training data used to develop the LLMs. Without detailed knowledge of the datasets and the specific programming tasks the LLMs were exposed to during training, it is difficult to ascertain whether the models have been inadvertently trained on similar tasks or even specific examples present in our evaluation dataset. This could result in biased performance metrics that do not accurately reflect the models' generalization capabilities.

Additionally, the evaluation process itself could introduce biases. For example, the selection of faulty programs and the criteria for what constitutes a "similar" program may affect the outcomes. We have attempted to mitigate this by using a random sampling approach and defining clear criteria for similarity, but subjective decisions are unavoidable.

The cost and accessibility of running extensive evaluations with LLMs also present equity and inclusion concerns. Not all researchers or educators may have the resources to replicate our study, which could limit the generalizability and applicability of our findings across different contexts.


\section{Citations}

\section{Appendices}

If your work needs an appendix, add it before the
``\verb|\end{document}|'' command at the conclusion of your source
document.

Start the appendix with the ``\verb|appendix|'' command:
\begin{verbatim}
  \appendix
\end{verbatim}
and note that in the appendix, sections are lettered, not
numbered. This document has two appendices, demonstrating the section
and subsection identification method.


%%
%% The next two lines define the bibliography style to be used, and
%% the bibliography file.
\bibliographystyle{ACM-Reference-Format}
\bibliography{sample-base}


%%
%% If your work has an appendix, this is the place to put it.
\appendix

\section{Research Methods}


\end{document}
\endinput
%%
%% End of file `sample-sigconf-authordraft.tex'.
